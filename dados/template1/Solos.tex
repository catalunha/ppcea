\documentclass[12pt,a4paper,twoside]{report}
%+++ PACOTES
%\usepackage[portuges]{babel} 
\usepackage[utf8]{inputenc} 
%--- PACOTES
%+++ LAYOUT DA PAGINA
\setlength{\topmargin}{1.0cm} 
\setlength{\headheight}{0.0cm} 
\setlength{\headsep}{0.5cm} 
\setlength{\textheight}{26cm} 
\setlength{\oddsidemargin}{1.0cm} 
\setlength{\evensidemargin}{1.0cm} 
\setlength{\textwidth}{19cm} 
\setlength{\footskip}{1cm} 
\setlength{\columnsep}{.5cm} 
\addtolength{\oddsidemargin}{-1in} 
\addtolength{\evensidemargin}{-1in} 
\addtolength{\topmargin}{-1in}
%--- LAYOUT DA PAGINA
\begin{document}


Disciplina: Solos

Código: CAG150

Período: 3

Categoria: obrigatoria

CH Teórica: 45

CH Prática: 15




Pré-requisito:
\begin{enumerate}
\end{enumerate}

Ementa:
\begin{enumerate}
\item Crescimento populacional e solo
\item Funções do solo
\item Fatores e processos de formação do solo
\item Atributos diagnóstico, horizontes e classificação do solo
\item Características físicas do solo
\item Propriedades químicas do solo
\item Acidez e correção do solo
\item Fertilidade do solo
\item Matéria orgânica do solo
\item Análise de solo
\item Interpretação de análise do solo
\item Importância do solo no ciclo de alguns elementos minerais
\item Manejo e Conservação do solo
\end{enumerate}



Bibliografia:


\begin{tabular}{llllp{8cm}}
Qde & Tipo & Uso & isbn-issn & Citação \\
8&livro&basica&978-85-7133-064-1&OLIVEIRA, J.B. Pedologia aplicada.  3ed. Piracicaba: FEALQ, 2008.\\
8&livro&basica&978-85-86504-04-4&MELO, V. F.; ALLEONI, L. R. F. (editores). Química e mineralogia do solo. Vol 1.Viçosa – MG: SBCS. 2009\\
8&livro&basica&978-85-86504-05-1&MELO, V. F.; ALLEONI, L. R. F. (editores). Química e mineralogia do solo. Vol 2.Viçosa – MG: SBCS. 2009\\
5&livro&complementar&978-85-7975-008-3&LEPSCH, I. F. Formação e Conservação do solo. São Paulo: Oficina de Textos, 2002\\
5&livro&complementar&978-85-204-1773-6&REICHARDT, K; TIMM, L. C. Solo, planta e atmosfera: conceitos, processos e aplicações. Barueri/SP: Manole, 2004.\\
2&livro&complementar&978-85-7025603-5&GLIESSMAN, S. R. Agroecologia: processos ecológicos em agricultura sustentável. 2ed. Porto Alegre: Ed.Universidade/UFRGS. 2001\\
2&livro&complementar&&BRADY, N. C. Natureza e propriedades dos solos. Trad. Antônio B. N. Figueiredo. 7ed. Rio de Janeiro: Freitas Bastos, 1989.\\
\end{tabular}

\end{document}
    