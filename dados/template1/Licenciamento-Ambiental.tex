\documentclass[12pt,a4paper,twoside]{report}
%+++ PACOTES
%\usepackage[portuges]{babel} 
\usepackage[utf8]{inputenc} 
%--- PACOTES
%+++ LAYOUT DA PAGINA
\setlength{\topmargin}{1.0cm} 
\setlength{\headheight}{0.0cm} 
\setlength{\headsep}{0.5cm} 
\setlength{\textheight}{26cm} 
\setlength{\oddsidemargin}{1.0cm} 
\setlength{\evensidemargin}{1.0cm} 
\setlength{\textwidth}{19cm} 
\setlength{\footskip}{1cm} 
\setlength{\columnsep}{.5cm} 
\addtolength{\oddsidemargin}{-1in} 
\addtolength{\evensidemargin}{-1in} 
\addtolength{\topmargin}{-1in}
%--- LAYOUT DA PAGINA
\begin{document}


Disciplina: Licenciamento Ambiental

Código: Nova001

Período: 8

Categoria: obrigatoria

CH Teórica: 45

CH Prática: 15




Pré-requisito:
\begin{enumerate}
\item Ecologia
\end{enumerate}

Ementa:
\begin{enumerate}
\item Licenciamento Ambiental – LA: conceitos e definições
\item Instrumentos de identificação e análise em LA
\item Identificação e caracterização de empreendimentos poluidores e passiveis de licenciamento ambiental
\item Legislação Pertinente ao LA
\item COEMA: Outorga de água, Reserva Legal e LA
\item Análise e Elaboração de Termos de Referência (Federal, Estadual e Municipal); considerando as particularidades do empreendimento (0pequeno e grande porte)
\item Etapas de elaboração e aprovação de um documento de licenciamento ambiental
\item Diferentes tipos de documentos de LA
\item Equipe elaboradora
\item Audiências públicas: atores sociais e tomada de decisão
\item Elaboração de parecer técnico de Estudos para licenciamento ambiental
\item Estudos de Caso
\end{enumerate}



Bibliografia:


\begin{tabular}{llllp{8cm}}
Qde & Tipo & Uso & isbn-issn & Citação \\
\end{tabular}

\end{document}
    