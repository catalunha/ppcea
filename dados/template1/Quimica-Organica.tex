\documentclass[12pt,a4paper,twoside]{report}
%+++ PACOTES
%\usepackage[portuges]{babel} 
\usepackage[utf8]{inputenc} 
%--- PACOTES
%+++ LAYOUT DA PAGINA
\setlength{\topmargin}{1.0cm} 
\setlength{\headheight}{0.0cm} 
\setlength{\headsep}{0.5cm} 
\setlength{\textheight}{26cm} 
\setlength{\oddsidemargin}{1.0cm} 
\setlength{\evensidemargin}{1.0cm} 
\setlength{\textwidth}{19cm} 
\setlength{\footskip}{1cm} 
\setlength{\columnsep}{.5cm} 
\addtolength{\oddsidemargin}{-1in} 
\addtolength{\evensidemargin}{-1in} 
\addtolength{\topmargin}{-1in}
%--- LAYOUT DA PAGINA
\begin{document}


Disciplina: Química Orgânica

Código: CAG123

Período: 3

Categoria: obrigatoria

CH Teórica: 45

CH Prática: 15




Pré-requisito:
\begin{enumerate}
\item Hidrologia
\end{enumerate}

Ementa:
\begin{enumerate}
\item Revisão sobre nomenclatura, aspectos estruturais e propriedades físicas de hidrocarbonetos, compostos aromáticos, alcoóis, fenóis, éteres, haletos orgânicos e compostos carbonílicos.
\item Alcanos e Cicloalcanos. Análise conformacional dos alcanos cíclicos e acíclicos.
\item Estereoquímica.
\item Reações de Adição a alcenos.
\item Substituição Eletrofílica Aromática.
\item Principais Reações de álcoois e haletos de Alquila.
\item Aldeídos e cetonas: Adição Nucleofílica à Carbonila
\item Ácidos carboxílicos e seus derivados. Adição nucleofílica acílica
\item Reações com aminas.
\end{enumerate}



Bibliografia:


\begin{tabular}{llllp{8cm}}
Qde & Tipo & Uso & isbn-issn & Citação \\
22&livro&basica&&SOLOMONS, T. W. G., FRYHLE, C. B.; Química Orgânica, Editora LTC, Vol 2, 10ª Edição, Rio de Janeiro, 2012.\\
22&livro&basica&&SOLOMONS, T. W. G., FRYHLE, C. B.; Química Orgânica, Editora LTC, Vol 1, 10ª Edição, Rio de Janeiro, 2012.\\
22&livro&basica&978-85-8055-053-5&CARREY, F. A.; Química Orgânica, 7º Edição, Editora Bookman, Vol. 1, 2011.\\
22&livro&basica&978-85-63308-89-4&CARREY, F. A.; Química Orgânica, 7º Edição, Editora Bookman, Vol. 2, 2011.\\
22&livro&basica&&BRUICE, P. Y.; Química Orgânica, Editora Pearson Prentice Hall, Vol 1, 4º Edição, 2006.\\
22&livro&basica&&BRUICE, P. Y.; Química Orgânica, Editora Pearson Prentice Hall, Vol 2, 4º Edição, 2006.\\
10&livro&complementar&&McMURRY, J.; Química Orgânica COMBO, Tradução da 6º Edição Norte-Americana, Editora Thomson, 2005.\\
\end{tabular}

\end{document}
    