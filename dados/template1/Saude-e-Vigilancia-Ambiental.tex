\documentclass[12pt,a4paper,twoside]{report}
%+++ PACOTES
%\usepackage[portuges]{babel} 
\usepackage[utf8]{inputenc} 
%--- PACOTES
%+++ LAYOUT DA PAGINA
\setlength{\topmargin}{1.0cm} 
\setlength{\headheight}{0.0cm} 
\setlength{\headsep}{0.5cm} 
\setlength{\textheight}{26cm} 
\setlength{\oddsidemargin}{1.0cm} 
\setlength{\evensidemargin}{1.0cm} 
\setlength{\textwidth}{19cm} 
\setlength{\footskip}{1cm} 
\setlength{\columnsep}{.5cm} 
\addtolength{\oddsidemargin}{-1in} 
\addtolength{\evensidemargin}{-1in} 
\addtolength{\topmargin}{-1in}
%--- LAYOUT DA PAGINA
\begin{document}

Disciplina: Saúde e Vigilância Ambiental

Pré-requisito:
\begin{enumerate}
\end{enumerate}

Ementa:
\begin{enumerate}
\item Apresentar e elaborar conceitos básicos em vigilância à saúde, especialmente conceitos de saúde, doença, risco, perigo e vigilância voltados ao estudo da questão saúde humana e relação ambiente-saúde humana
\item Propor, pesquisar e elaborar modelos, protocolos e exemplos de estudos em indicadores ambientais de saúde
\item Diagnóstico de relação ambiente-saude (baseado em modelos DPSIR ou FMPEEEA)
\item Avaliação de impactos a saúde (HIA)
\item Avaliação de risco à saúde (ARS).
\end{enumerate}

Bibliografia:
\begin{tabular}{lllll}
Qde & Tipo & Uso & isbn-issn & Citação \\
5&livro&basica&85-85676-56-6&Ogenis Magno Brilhante e Luiz Querino de A. Caldas (Coords.) Gestão e Avaliação de Risco em Saúde Ambiental. ISBN 85-85676-56-6. 2a reimpressão: 2004. 1a reimpressão: 2002. (1a edição: 1999). 155p.  Editora Fiocruz\\
5&livro&basica&9788538802198&Papini S. Vigilância em Saúde Ambiental - Uma Nova Área da Ecologia.2ª ed. revista e ampliada. Rio de Janeiro: Editora Atheneu; 2012\\
5&livro&basica&8520421881 (ISBN-13: 9788520421888)&Philippi Junior, Arlindo. Saneamento, saude e ambiente, fundamentos para um desenvolvimento sustentavel. ISBN: 8520421881 ISBN-13: 9788520421888 Editora: MANOLE\\
1&livro&complementar&8574197246&Luís Sérgio Ozório Valentim. Requalificação urbana, contaminação do solo e riscos à saúde: um caso na cidade de São Paulo. Annablume, 2007. 159 P.\\
1&livro&complementar&85-85471-11-5&Minayo MCS, Miranda AC. Saúde e ambiente sustentável: estreitando os nós. Rio de Janeiro: Ed. Fiocruz/Abrasco; 2002.\\
1&livro&complementar&85-85676-72-8. 2000.&Freitas CM, Porto MFS, Machado JMH, organizadores. Acidentes industriais ampliados: desafios e perspectivas para o controle e a prevenção. Rio de Janeiro: Editora Fiocruz; 2000.\\
5&livro&complementar&9788527106856&AKERMAN, Marco . Saúde e desenvolvimento local: princípios, conceitos, práticas e cooperação técnica. São Paulo: Hucitec, 2005.\\
1&livro&complementar&85-85676-80-9&Cristina Lucia Silveira Sisinno, Rosália Maria de Oliveira (oRG.) Resíduos Sólidos, Ambiente e Saúde: uma visão multidisciplinar. 3ª reimpressão: 2006. 2ª reimpressão: 2003. 1ª reimpressão: 2002 (1ª edição: 2000). 138 P ISBN: 85-85676-80-9\\
\end{tabular}

\end{document}
    