\documentclass[12pt,a4paper,twoside]{report}
%+++ PACOTES
%\usepackage[portuges]{babel} 
\usepackage[utf8]{inputenc} 
%--- PACOTES
%+++ LAYOUT DA PAGINA
\setlength{\topmargin}{1.0cm} 
\setlength{\headheight}{0.0cm} 
\setlength{\headsep}{0.5cm} 
\setlength{\textheight}{26cm} 
\setlength{\oddsidemargin}{1.0cm} 
\setlength{\evensidemargin}{1.0cm} 
\setlength{\textwidth}{19cm} 
\setlength{\footskip}{1cm} 
\setlength{\columnsep}{.5cm} 
\addtolength{\oddsidemargin}{-1in} 
\addtolength{\evensidemargin}{-1in} 
\addtolength{\topmargin}{-1in}
%--- LAYOUT DA PAGINA
\begin{document}


Disciplina: Resistência dos Materiais

Código: ENG071

Período: 4

Categoria: obrigatoria

CH Teórica: 60

CH Prática: 0




Pré-requisito:
\begin{enumerate}
\item Solos
\end{enumerate}

Ementa:
\begin{enumerate}
\item Princípios gerais. Vínculos Estruturais.Estrutura
\item Equilibrio de Forças e Momentos.Força Normal
\item Carga Distribuída. Linha de ação da Resultante
\item Tração e Compressão. Tensão Normal. Lei de Hooke. Dimensionamento
\item Treliças Planas. Método dos nós e de Ritter. Dimensionamento
\item Cisalhamento Puro. Tensão de Cisalhamento. Pressão de contato.Dimensionamento
\item Força Cortante e Momento Fletor. Diagramas
\item Flexão. Tensão Normal na Flexão. Dimensionamento na Flexão
\item Torção. Potência. Tensaõ de Cisalhamento na Torção. Distorção. Ângulo de Torção. Dimensionamento de Eixos-Árvore
\item Flambagem. Carga Crítica. Indice de esbeltez. Tensão crítica. Normas. Dimensionamento e verificação estrutural
\end{enumerate}



Bibliografia:


\begin{tabular}{llllp{8cm}}
Qde & Tipo & Uso & isbn-issn & Citação \\
8&livro&basica&&MELCONIAN, S.. Mecânica técnica e Resistência dos Materiais. 9. ed., rev. atualizada. São Paulo: Érica. 2013\\
8&livro&basica&&BERR, F.P.; JOHNSTON, E. R. Resistência dos Materiais. São Paulo, Makron Books, 2010. GARE, J. M. Mecânica dos Materiais. São Paulo: Pioneira Thomson Learning, 2011. 698 p.\\
8&livro&basica&&HIBBELER, R. C.. Resistência dos Materiais. 3ª ed. Rio de Janeiro: LTC, 2000.\\
6&livro&basica&&CHIAVERINI, V. (1986). Tecnologia Mecânica, vol. I, II e III, 2ª ed. Brasil, S. Paulo: Editora McGraw-Hill.\\
4&livro&complementar&&NASH, W.A. Resistência dos Materiais. São Paulo, Mc Graw Hill, 1982.\\
4&livro&complementar&&TIMOSHENKO, S. P.. Mecânica dos sólidos. Rio de Janeiro: LTC. 1989.\\
\end{tabular}

\end{document}
    