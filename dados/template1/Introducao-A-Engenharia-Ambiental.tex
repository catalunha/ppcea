\documentclass[12pt,a4paper,twoside]{report}
%+++ PACOTES
%\usepackage[portuges]{babel} 
\usepackage[utf8]{inputenc} 
%--- PACOTES
%+++ LAYOUT DA PAGINA
\setlength{\topmargin}{1.0cm} 
\setlength{\headheight}{0.0cm} 
\setlength{\headsep}{0.5cm} 
\setlength{\textheight}{26cm} 
\setlength{\oddsidemargin}{1.0cm} 
\setlength{\evensidemargin}{1.0cm} 
\setlength{\textwidth}{19cm} 
\setlength{\footskip}{1cm} 
\setlength{\columnsep}{.5cm} 
\addtolength{\oddsidemargin}{-1in} 
\addtolength{\evensidemargin}{-1in} 
\addtolength{\topmargin}{-1in}
%--- LAYOUT DA PAGINA
\begin{document}


Disciplina: Introdução A Engenharia Ambiental

Código: Nova011

Período: 1

Categoria: obrigatoria

CH Teórica: 30

CH Prática: 0




Pré-requisito:
\begin{enumerate}
\item Cálculo Diferencial e Integral II
\end{enumerate}

Ementa:
\begin{enumerate}
\item Engenharia ambiental: perfil profissional e mercado de trabalho.
\item Histórico do movimento ambientalista.
\item Noções gerais de poluição ambiental e suas interferências no meio ambiente.
\item Noções gerais de degradação ambiental e suas interferências no meio ambiente.
\item Introdução ao saneamento básico: Água
\item Introdução ao saneamento básico: Esgoto
\item Introdução ao saneamento básico: Resíduos Sólidos
\item Introdução ao saneamento básico: Drenagem Urbana
\item Introdução ao Licenciamento Ambiental
\item Noções de Legislação Ambiental
\end{enumerate}



Bibliografia:


\begin{tabular}{llllp{8cm}}
Qde & Tipo & Uso & isbn-issn & Citação \\
1&livro&basica&&BRAGA, B. et al. Introdução a Engenharia Ambiental. São Paulo: Ed. Prentice Hall, 2002.\\
1&livro&basica&&MOTA, S. Introdução à Engenharia Ambiental. Rio de Janeiro: Ed. ABES, 2000.\\
0&livro&complementar&9788528608021&GUERRA, Antonio José Teixeira; CUNHA, Sandra Baptista da (Org.). Impactos ambientais urbanos no Brasil. 6.ed. Rio de Janeiro: Bertrand Brasil, 2010. 416 p.\\
0&livro&complementar&8512490403&FELLENBERG, Gunter. Introdução aos problemas da poluição ambiental. São Paulo, SP: EPU: 1980. xvi, 196 p.\\
1&livro&complementar&&BRASIL. Fundação Nacional de Saúde. Manual de Saneamento. 3a ed. Brasília: Fundação Nacional de Saúde, 2016. 408 p.\\
\end{tabular}

\end{document}
    