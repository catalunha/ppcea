\documentclass[12pt,a4paper,twoside]{report}
%+++ PACOTES
%\usepackage[portuges]{babel} 
\usepackage[utf8]{inputenc} 
%--- PACOTES
%+++ LAYOUT DA PAGINA
\setlength{\topmargin}{1.0cm} 
\setlength{\headheight}{0.0cm} 
\setlength{\headsep}{0.5cm} 
\setlength{\textheight}{26cm} 
\setlength{\oddsidemargin}{1.0cm} 
\setlength{\evensidemargin}{1.0cm} 
\setlength{\textwidth}{19cm} 
\setlength{\footskip}{1cm} 
\setlength{\columnsep}{.5cm} 
\addtolength{\oddsidemargin}{-1in} 
\addtolength{\evensidemargin}{-1in} 
\addtolength{\topmargin}{-1in}
%--- LAYOUT DA PAGINA
\begin{document}

Disciplina: Eletricidade aplicada e Instrumentação Ambiental

Pré-requisito:
\begin{enumerate}
\end{enumerate}

Ementa:
\begin{enumerate}
\item Fundamentos da energia elétrica e a instrumentação ambiental
\item Planejamento e Projeto de instalações elétricas
\item Dimensionamento de condutores, eletrodutos e proteção
\item Motores elétricos.
\item Variáveis ambientais e sua medição
\item Mecânica e Eletrônica dos instrumentos
\item Medição e armazenamento de dados ambientais
\end{enumerate}

Bibliografia:
\begin{tabular}{lllll}
Qde & Tipo & Uso & isbn-issn & Citação \\
10&livro&basica&978-85-7605-208-1&Cotrim, Ademaro A.M.B., 1939-. Instalações elétricas. 5ed. São Paulo. Pearson Pretice Hall. 2009. 08-10784. CDD 621.3192.  UFT00067178\\
10&livro&basica&978852161520-0&Instalacoes eletricas industriais. Mamede Filho, Joao. LTC Ed., 7.ed. de acordo com a nova norma brasileira, NBR 5410:2004 e 14.039. 2009. CDD: 621.31924\\
10&livro&basica&9788521615675&Instalações elétricas. Creder, Helio.  LTC,  15. ed. 2007. CDD: 621.31042\\
10&livro&basica&9788521615897&Instalações elétricas. Niskier, Julio.  LTC, 5. ed. 2008. CDD: 621.31042\\
10&livro&basica&UFT00087969&Instrumentacao industrial Interciencia, 2 ed. 583 p.\\
10&livro&basica&UFT00067310&Instrumentacao, controle e automacao de processos Alves, Jose Luiz Loureiro. LTC Ed., 270p.\\
10&livro&basica&UFT00075461&Circuitos experimentais em eletricidade e eletrônica Tucci, Wilson J. Nobel, il.\\
\end{tabular}

\end{document}
    