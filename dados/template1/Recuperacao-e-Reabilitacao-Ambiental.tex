\documentclass[12pt,a4paper,twoside]{report}
%+++ PACOTES
%\usepackage[portuges]{babel} 
\usepackage[utf8]{inputenc} 
%--- PACOTES
%+++ LAYOUT DA PAGINA
\setlength{\topmargin}{1.0cm} 
\setlength{\headheight}{0.0cm} 
\setlength{\headsep}{0.5cm} 
\setlength{\textheight}{26cm} 
\setlength{\oddsidemargin}{1.0cm} 
\setlength{\evensidemargin}{1.0cm} 
\setlength{\textwidth}{19cm} 
\setlength{\footskip}{1cm} 
\setlength{\columnsep}{.5cm} 
\addtolength{\oddsidemargin}{-1in} 
\addtolength{\evensidemargin}{-1in} 
\addtolength{\topmargin}{-1in}
%--- LAYOUT DA PAGINA
\begin{document}


Disciplina: Recuperação e Reabilitação Ambiental

Código: CAG279

Período: 9

Categoria: obrigatoria

CH Teórica: 30

CH Prática: 30




Pré-requisito:
\begin{enumerate}
\item Avaliação de Impactos Ambientais
\end{enumerate}

Ementa:
\begin{enumerate}
\item Conceituação
\item Caracterização de áreas degradadas
\item Recuperação e restauração de ambientes degradados
\item Fontes e efeitos da degradação de ambientes considerando os diferentes compartimentos ambientais
\item Graus de degradação e funções dos compartimentos ambientais
\item Objetivos da recuperação ou restauração de ambientes degradados (RAD)
\item Métodos e aplicações da Engenharia Convencional e Reabilitação de ambientes degradados: Escolha de métodos para diferentes graus de degradação
\item Princípios de bioengenharia aplicada aos processos de RAD
\item Métodos e uso da fitorremediação e biorremediação em ambientes degradados.; funcionalidade e escolha de espécies vegetais na RAD
\item Recomposição de matas ripárias e corredores ecológicos
\item Sistemas agroflorestais como instrumento de RAD
\item Monitoramento de processos de RAD
\item Atividades potencialmente degradantes: estudos de caso.
\end{enumerate}



Bibliografia:


\begin{tabular}{llllp{8cm}}
Qde & Tipo & Uso & isbn-issn & Citação \\
\end{tabular}

\end{document}
    