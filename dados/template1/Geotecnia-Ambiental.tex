\documentclass[12pt,a4paper,twoside]{report}
%+++ PACOTES
%\usepackage[portuges]{babel} 
\usepackage[utf8]{inputenc} 
%--- PACOTES
%+++ LAYOUT DA PAGINA
\setlength{\topmargin}{1.0cm} 
\setlength{\headheight}{0.0cm} 
\setlength{\headsep}{0.5cm} 
\setlength{\textheight}{26cm} 
\setlength{\oddsidemargin}{1.0cm} 
\setlength{\evensidemargin}{1.0cm} 
\setlength{\textwidth}{19cm} 
\setlength{\footskip}{1cm} 
\setlength{\columnsep}{.5cm} 
\addtolength{\oddsidemargin}{-1in} 
\addtolength{\evensidemargin}{-1in} 
\addtolength{\topmargin}{-1in}
%--- LAYOUT DA PAGINA
\begin{document}

Disciplina: Geotecnia Ambiental

Pré-requisito:
\begin{enumerate}
\end{enumerate}

Ementa:
\begin{enumerate}
\item Interação solo contaminante
\item Transporte de contaminantes no solo
\item Amostragem do solo e água subterrânea
\item Valores norteadores de nível de contaminação no solo e na água subterrânea
\item Medidas de intervenção no gerenciamento de áreas contaminadas
\item Técnicas de contenção e remediação de solo e água subterrânea
\item Barragens e dragagem: tipos, operação e aspectos ambientais
\end{enumerate}

Bibliografia:
\begin{tabular}{lllll}
Qde & Tipo & Uso & isbn-issn & Citação \\
8&livro&basica&978-85-862-3873-4&BOSCOV, M.E.G. Geotecnia Ambiental. 1. Ed. São Paulo : Oficina de Textos, 2008. 248 p.\\
10&livro&basica&978-85-352-8058-6&ZUQUETTE, L.V. Geotecnia Ambiental. 1. Ed. São Paulo : Elsevier , 2015. 432 p.\\
5&livro&basica&978-85-727-000-2&OLIVEIRA, A. M.S., BRITO, S.N.A. Geologia de Engenharia. São Paulo : ABGE, 1998. 587 p.\\
3&livro&complementar&978-85-212-0956-0&QUEIROZ, RUDNEY C. Geologia e Geotecnia Básica para Engenharia Civil. 1. ed. São Paulo : Bluncher, 2016. 416 p.\\
2&livro&complementar&978-85-847-8031-6.&BATES, J.  Barragens de Rejeitos. São Paulo: Signus Editora, 2002. 112 p.\\
0&livro&complementar&978-85-09-00179-7&MORAES, S.L.; TEIXEIRA, C.E.; MAXIMIANO, A.M.S. Guia de elaboração de planos de intervenção para o gerenciamento de áreas contaminadas. 1. ed. rev. São Paulo : IPT, 2014.395 p. Disponível em: https://www.google.com.br/#q=GERENCIAMENTO+DE+%C3%81REAS+CONTAMINADAS+IPT. Acesso em: 17/01/2017.\\
2&livro&complementar&&SCHIANETZ, B. Passivos Ambientais: levantamento histórico da periculosidade: ações de recuperação. Curitiba : SENAI-PR, 1999\\
5&livro&complementar&978-85-797-5086-1&GIAMPÁ, C.E.Q.; GONÇALVES, V. G. Águas Subterrâneas e Poços Tubulares Profundos. 2. ed. São Paulo : Oficina de Textos, 2013. 496 p\\
\end{tabular}

\end{document}
    