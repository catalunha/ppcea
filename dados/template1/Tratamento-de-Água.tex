\documentclass[12pt,a4paper,twoside]{report}
%+++ PACOTES
%\usepackage[portuges]{babel} 
\usepackage[utf8]{inputenc} 
%--- PACOTES
%+++ LAYOUT DA PAGINA
\setlength{\topmargin}{1.0cm} 
\setlength{\headheight}{0.0cm} 
\setlength{\headsep}{0.5cm} 
\setlength{\textheight}{26cm} 
\setlength{\oddsidemargin}{1.0cm} 
\setlength{\evensidemargin}{1.0cm} 
\setlength{\textwidth}{19cm} 
\setlength{\footskip}{1cm} 
\setlength{\columnsep}{.5cm} 
\addtolength{\oddsidemargin}{-1in} 
\addtolength{\evensidemargin}{-1in} 
\addtolength{\topmargin}{-1in}
%--- LAYOUT DA PAGINA
\begin{document}

Disciplina: Tratamento de Água

Pré-requisito:
\begin{enumerate}
\item Geologia
\end{enumerate}

Ementa:
\begin{enumerate}
\item Concepção de Sistemas de Tratamento de Água em Função da Qualidade da Água Bruta
\item Projeto de ETAs de Ciclo Completo com Emprego da Decantação ou da Flotação por ar Dissolvido para Clarificação
\item Projeto de ETAs de Filtração Direta Descendente
\item Projeto de ETAs de Filtração Direta Ascendente
\item Projeto de ETAs de Dupla Filtração
\item Projeto de ETAs por Floto-Filtração
\item Projeto de ETAs de filtração em Múltiplas Etapas
\item Desinfecção para água de consumo humano
\item Adsorção em Carvão Ativado
\item Tratamento dos Resíduos Gerados nas ETAs e Reuso da Água Recuperada
\end{enumerate}

Bibliografia:
\begin{tabular}{lllll}
Qde & Tipo & Uso & isbn-issn & Citação \\
5&livro&basica&&DI BERNARDO, L.; DI BERNARDO, A. Métodos e Técnicas de Tratamento de Água. Volume I e II. Ed. RIMA. 2ª. Ed. 2005\\
5&livro&basica&&LIBANIO, Fundamentos de Qualidade e Tratamento de Água. 3ª Ed. Belo Horizonte. 2017\\
5&livro&basica&&DI BERNARDO, L., DANTAS, A D. B., VOLTAN, P. E. N. Tratabilidade de Água e dos resíduos gerados em estações de tratamento de água. São Carlos, LDiBe editora, 2011\\
\end{tabular}

\end{document}
    