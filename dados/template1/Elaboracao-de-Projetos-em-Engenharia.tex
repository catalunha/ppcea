\documentclass[12pt,a4paper,twoside]{report}
%+++ PACOTES
%\usepackage[portuges]{babel} 
\usepackage[utf8]{inputenc} 
%--- PACOTES
%+++ LAYOUT DA PAGINA
\setlength{\topmargin}{1.0cm} 
\setlength{\headheight}{0.0cm} 
\setlength{\headsep}{0.5cm} 
\setlength{\textheight}{26cm} 
\setlength{\oddsidemargin}{1.0cm} 
\setlength{\evensidemargin}{1.0cm} 
\setlength{\textwidth}{19cm} 
\setlength{\footskip}{1cm} 
\setlength{\columnsep}{.5cm} 
\addtolength{\oddsidemargin}{-1in} 
\addtolength{\evensidemargin}{-1in} 
\addtolength{\topmargin}{-1in}
%--- LAYOUT DA PAGINA
\begin{document}

Disciplina: Elaboração de Projetos em Engenharia

Pré-requisito:
\begin{enumerate}
\end{enumerate}

Ementa:
\begin{enumerate}
\item Gestão de projetos
\item Partes constituintes de um projeto de engenharia
\item Áreas de conhecimento em gerenciamento de projetos
\item Características do Gerente de Projeto
\item Fases e Ciclo de Vida de Um Projeto
\item Influências Organizacionais nos projetos
\item Definição e tipos de Projetos de engenharia
\item Linhas básicas da elaboração de uma proposta
\item Técnicas de elaboração e desenho de projetos de engenharia
\item Estratégias de captação de recursos financeiros
\item Oficina de projetos de engenharia
\end{enumerate}

Bibliografia:
\begin{tabular}{lllll}
Qde & Tipo & Uso & isbn-issn & Citação \\
0&livro&basica&9788521624004&Martland, Carl D. Avaliação de Projetos: Por uma Infraestrutura Sustentável. LTC:São Paulo. 1014. 424p.\\
0&livro&basica&521634455&ANDERSON, PHILIPKORTSCHOT, MARK T.MCCAHAN, SUSAN. PROJETOS DE ENGENHARIA - UMA INTRODUÇAO. LTC: São Paulo. 2014\\
11&livro&basica&978-85-7452-696-6&MASSARI. VICTOR L. Gerenciamento Ágil de Projetos. Brasport:2014. 256p.\\
0&livro&basica&9788502227101&Keelling,Ralph e Branco, Renato Henrique Ferreira. Gestão de Projetos - Uma Abordagem Global - 3ª Ed. 2014. 288p.\\
11&livro&complementar&9788550801711&Monteiro Teixeira, Júlio. Gestão Visual De Projetos - Utilizando A Informação Para Inovar. Saraiva: São Paulo. 2014. 209p.\\
11&livro&complementar&&Guerreiro, Fernando.FERRAMENTAS ESTRATÉGICAS NA GESTÃO DE PROJETOS. Saraiva:São Paulo. 2016. e-book.\\
0&livro&complementar&&Guide to the Project Management Body of Knowledge (PMBOK® Guide). PMI. 6 ed. 2017\\
0&outros&outro&&O que é Gerenciamento de Projetos? disponível em:<https://brasil.pmi.org/brazil/AboutUs/WhatIsProjectManagement.aspx>. Acesso em: 14 maio 2018.\\
\end{tabular}

\end{document}
    