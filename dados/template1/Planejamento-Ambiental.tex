\documentclass[12pt,a4paper,twoside]{report}
%+++ PACOTES
%\usepackage[portuges]{babel} 
\usepackage[utf8]{inputenc} 
%--- PACOTES
%+++ LAYOUT DA PAGINA
\setlength{\topmargin}{1.0cm} 
\setlength{\headheight}{0.0cm} 
\setlength{\headsep}{0.5cm} 
\setlength{\textheight}{26cm} 
\setlength{\oddsidemargin}{1.0cm} 
\setlength{\evensidemargin}{1.0cm} 
\setlength{\textwidth}{19cm} 
\setlength{\footskip}{1cm} 
\setlength{\columnsep}{.5cm} 
\addtolength{\oddsidemargin}{-1in} 
\addtolength{\evensidemargin}{-1in} 
\addtolength{\topmargin}{-1in}
%--- LAYOUT DA PAGINA
\begin{document}


Disciplina: Planejamento Ambiental

Código: CSA278

Período: 8

Categoria: obrigatoria

CH Teórica: 45

CH Prática: 15




Pré-requisito:
\begin{enumerate}
\end{enumerate}

Ementa:
\begin{enumerate}
\item Planejamento: conceitos, principios, tipos e fases do planejamento; A organização como sistema ; A mudança como parte da organização
\item Planejamento Estratégico institucional e Processo de Tomada de Decisão
\item Planejamento Ambiental: histórico, paradigmas, conceitos, estruturas e instrumentos
\item Planejamento Ambiental com ênfase em Bacias Hidrográficas
\item Plano de manejo de Unidades de Conservação
\item Planejamento como suporte a gestão de conflitos socioambientais e Metodologias participativas;
\item Planejamento Urbano e Plano diretor municipal,
\item Zoneamento como Instrumento de Planejamento e Gestão Ambiental (Zoneamento Agrícola e Ambiental e  Zoneamento Econômico Ecológico – ZEE)
\item Planejamento no contexto das Mudanças Climáticas e Mercado de Carbono
\end{enumerate}



Bibliografia:


\begin{tabular}{llllp{8cm}}
Qde & Tipo & Uso & isbn-issn & Citação \\
10&livro&basica&&OLIVEIRA, F. E. M. de. Estatística e Probabilidade. São Paulo: Ed. Atlas, 221p. 1999.\\
\end{tabular}

\end{document}
    