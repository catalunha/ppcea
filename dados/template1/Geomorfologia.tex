\documentclass[12pt,a4paper,twoside]{report}
%+++ PACOTES
%\usepackage[portuges]{babel} 
\usepackage[utf8]{inputenc} 
%--- PACOTES
%+++ LAYOUT DA PAGINA
\setlength{\topmargin}{1.0cm} 
\setlength{\headheight}{0.0cm} 
\setlength{\headsep}{0.5cm} 
\setlength{\textheight}{26cm} 
\setlength{\oddsidemargin}{1.0cm} 
\setlength{\evensidemargin}{1.0cm} 
\setlength{\textwidth}{19cm} 
\setlength{\footskip}{1cm} 
\setlength{\columnsep}{.5cm} 
\addtolength{\oddsidemargin}{-1in} 
\addtolength{\evensidemargin}{-1in} 
\addtolength{\topmargin}{-1in}
%--- LAYOUT DA PAGINA
\begin{document}


Disciplina: Geomorfologia

Código: CET099

Período: 2

Categoria: obrigatoria

CH Teórica: 30

CH Prática: 15




Pré-requisito:
\begin{enumerate}
\item Sistemas de Gestão Ambiental
\end{enumerate}

Ementa:
\begin{enumerate}
\item Processos geomorfológicos exógenos
\item Solos: parâmetros físicos, permeabilidade e compactação
\item Investigação do solo e do subsolo
\item Ambientes geológicos de erosão e deposição
\item Movimento de Massa
\item Bases do Mapeamento Geomorfológico
\item Introdução à Geomorfologia do Brasil e do Tocantins.
\end{enumerate}



Bibliografia:


\begin{tabular}{llllp{8cm}}
Qde & Tipo & Uso & isbn-issn & Citação \\
10&livro&basica&978-85-212-0130-4&CHRISTOFOLETTI, A. Geomorfologia. 2. ed. São Paulo : Blucher, 1980. 188p.\\
5&livro&basica&978-85-862-3865-9&FLORENZANO, T. G. Geomorfologia: Conceitos e tecnologias atuais. 1. ed. São Paulo : Oficina de Textos, 2008. 320 p.\\
5&livro&basica&978-85-286-0326-2&GUERRA, A.T.J. Geomorfologia – uma atualização de bases e conceitos. 9. ed. Rio de Janeiro: Bertrand Brasil, 2012. 472 p.\\
5&livro&complementar&978-85-040-1439-6&TEIXEIRA, W.; TOLEDO, M. C. M.; FAIRCHILD, T. R.; TAIOLI, F. Decifrando a Terra. 2. ed. São Paulo : Oficina de Textos, 2009. 624 p.\\
5&livro&complementar&978-85-212-0186-9&GUIDICINI, G; NIABLE, C.N. Estabilidade de Taludes Naturais e de Escavação. 2. ed. São Paulo: Blucher, 1984. 216 p.\\
3&livro&complementar&978-85-352-5954-4&ASHBY,M.F. Engenharia ambiental: conceitos, tecnologia e gestão. 1. ed. Rio de Janeiro : Elsevier, 2013. 789 p.\\
3&livro&complementar&978-85-797-5079-3&GUERRA, A. J. T; JORGE, M. C. O. Processos erosivos e recuperação de áreas degradadas. 1. ed. São Paulo: Oficina de Textos, 2013. 192 p.\\
4&livro&complementar&978-85-862-3851-2&PINTO, C. S. Curso Básico de Mecânica dos Solos. 3. ed. São Paulo: Oficina de Textos, 2006. 367 p.\\
0&outros&outro&&CASSETI, V. Geomorfologia. [S.l.]: [2005]. Disponível em: <http://www.funape.org.br/geomorfologia/>. Acesso em: . 17/01/2017.\\
0&outros&outro&978-85-240-4110-5&IBGE. Manual técnico de geomorfologia. 2. ed. Rio de Janeiro : IBGE, 2009. Disponível em: ftp://geoftp.ibge.gov.br/documentos/recursos_naturais/manuais_tecnicos/manual_tecnico_geomorfologia.pdf. Acesso em: 17/01/2017\\
\end{tabular}

\end{document}
    