\documentclass[12pt,a4paper,twoside]{report}
%+++ PACOTES
%\usepackage[portuges]{babel} 
\usepackage[utf8]{inputenc} 
%--- PACOTES
%+++ LAYOUT DA PAGINA
\setlength{\topmargin}{1.0cm} 
\setlength{\headheight}{0.0cm} 
\setlength{\headsep}{0.5cm} 
\setlength{\textheight}{26cm} 
\setlength{\oddsidemargin}{1.0cm} 
\setlength{\evensidemargin}{1.0cm} 
\setlength{\textwidth}{19cm} 
\setlength{\footskip}{1cm} 
\setlength{\columnsep}{.5cm} 
\addtolength{\oddsidemargin}{-1in} 
\addtolength{\evensidemargin}{-1in} 
\addtolength{\topmargin}{-1in}
%--- LAYOUT DA PAGINA
\begin{document}

Disciplina: Projeto e Gestão dos Resíduos Sólidos

Pré-requisito:
\begin{enumerate}
\end{enumerate}

Ementa:
\begin{enumerate}
\item Introdução aos Resíduos Sólidos.
\item Caracterização, levantamentos de dados e preparo de amostras;
\item Definição de Resíduos Sólidos;
\item Política nacional de resíduos Sólidos
\item 3R: Reciclagem, redução e reutilização;
\item Limpeza Pública;
\item Gerenciamento de Resíduos Sólidos: acondicionamento, coleta, transporte, transferência dos Resíduos;
\item Principais formas de tratamento e disposição final dos resíduos sólidos: lixões, aterro sanitário; Projeto das unidades
\item Tratamento Térmico: incineração, Pirólise, autoclavagem e microondas;
\item Compostagem, Resíduos Perigosos. Projeto de uma unidade de compostagem
\end{enumerate}

Bibliografia:
\begin{tabular}{lllll}
Qde & Tipo & Uso & isbn-issn & Citação \\
3&livro&basica&&D’ALMEIDA, Maria Luiza Otero, VILHENA, André. Lixo Municipal: Manual de Gerenciamento Integrado. São Paulo: IPT/CEMPRE, 2000.\\
0&livro&basica&&BRASIL. Fundação Nacional de Saúde. Manual de Saneamento. 3a ed. Brasília: Fundação Nacional de Saúde, 2016. 408 p.\\
1&juridico&basica&&BRASIL. Lei 12305. Política Nacional de Resíduos Sólidos\\
0&livro&basica&&MONTEIRO, J. H. P. et al. Manual de gerenciamento integrado de resíduos sólidos. Rio de Janeiro, IBAM, 2001. 200 p. Disponível em: < www.web-resol.org/cartilha4/manual.pdf>. Acesso em 30 ago. 2010.\\
1&juridico&complementar&&ASSOCIAÇÃO BRASILEIRA DE NORMAS TÉCNICAS-ABNT. NBR 13896 - Aterros de resíduos não perigosos: critérios para projeto, implantação e operação. Rio de Janeiro, 1997.\\
0&juridico&complementar&&ASSOCIAÇÃO BRASILEIRA DE NORMAS TÉCNICAS-ABNT. NBR 8419 - Apresentação de projetos de aterros sanitários de resíduos sólidos urbanos - Procedimento. Rio de Janeiro, 1996.\\
0&livro&complementar&&PEREIRA NETO, João Tinôco. Manual de compostagem: processo de baixo custo. 1. ed. Viçosa, MG: Ed. da UFV, 2007. 81 p. (Soluções).\\
\end{tabular}

\end{document}
    