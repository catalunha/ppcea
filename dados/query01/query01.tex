
\documentclass[12pt,a4paper,twoside]{report}
%+++ PACOTES
%\usepackage[portuges]{babel} 
\usepackage[utf8]{inputenc} 
\usepackage{fancyhdr} 
%\usepackage{listings} 
%\usepackage{lastpage} 
\usepackage{multirow,colortbl,array} 
%\usepackage{graphicx} 
%\usepackage{amssymb} 
%\usepackage{amsmath} 
%\usepackage{multicol}
%\usepackage{comment}
%\usepackage{ifthen} 
%\usepackage{pdfpages}
%\usepackage{datetime}
%\usepackage{colortbl} 
\usepackage{longtable}
%--- PACOTES

%+++ LAYOUT DA PAGINA
\setlength{\topmargin}{1.0cm} 
\setlength{\headheight}{0.0cm} 
\setlength{\headsep}{0.5cm} 
\setlength{\textheight}{26cm} 
\setlength{\oddsidemargin}{1.0cm} 
\setlength{\evensidemargin}{1.0cm} 
\setlength{\textwidth}{19cm} 
\setlength{\footskip}{1cm} 
\setlength{\columnsep}{.5cm} 
\addtolength{\oddsidemargin}{-1in} 
\addtolength{\evensidemargin}{-1in} 
\addtolength{\topmargin}{-1in}
%--- LAYOUT DA PAGINA
\begin{document}
\begin{longtable}{l|l|p{4cm}|l|p{3cm}|p{3cm}|l|l}
id&tipo&citacao&id&nome&isbn\_issn&qde&uso\\
1&livro&Cotrim, Ademaro A.M.B., 1939-. Instalações elétricas. 5ed. São Paulo. Pearson Pretice Hall. 2009. 08-10784. CDD 621.3192.  UFT00067178&135&Eletricidade aplicada e Instrumentação Ambiental&978-85-7605-208-1&10&basica\\
2&livro&dasdasd asd&104&Administração e Empreendedorismo&11&1&basica\\
3&livro&Economia Ambiental - Gestão de custos e investimentos, 2011. Editora Del Rey.&125&Economia Ambiental&9788538401773&20&basica\\
4&livro&Manual para Valoração de Recursos Naturais, Ronaldo Seroa da Mota, 1997&125&Economia Ambiental&Não tem&0&basica\\
5&livro&Sano, S.M.; Almeida, S.P.; Ribeiro, J.F. (eds). Cerrado, Ecologia e Flora. Editora, Volume 1. Embrapa, Brasília, DF. 406p. 2008.&96&Caracterização Ambiental I&978-85-7383-397-3&10&basica\\
6&livro&Ricklefs, R.E. A Economia da Natureza. Editora Guanabara Koogan, Rio de Janeiro, RJ, sétima edição. 503p. 2016&96&Caracterização Ambiental I&9788527729628&10&basica\\
7&livro&Cullen Jr., L.; Rudran, R.; Valladares-Pádua, C. (orgs). Métodos de Estudos em Biologia da Conservação e Manejo da Vida Silvestre. Editora UFPR, segunda edição revisada. 652p. 2006.&96&Caracterização Ambiental I&978-8573351743&5&basica\\
8&livro&Odum, E.P. Fundamentos em Ecologia. Fundação  Calouste Gulbenkian. Lisboa, Portugal, sétima edição. 928p. 2004.&96&Caracterização Ambiental I&9789723101584&5&basica\\
9&livro&Cain, M.; Bowman, W.D.; Hacker, S.D. Ecologia. Artmed Editora. Porto Alegre, RS. 644p. 2011.&96&Caracterização Ambiental I&9788536325477&5&basica\\
10&livro&Townsend, C.R.; Begon, M. Harper, J.L. Fundamentos em Ecologia. Artmed Editora. Porto Alegre, RS. 576p. 2010&96&Caracterização Ambiental I&9788536320649&2&complementar\\
11&livro&Harper, J.L. Ecologia: De indivíduos a Ecossistemas. Editora Saraiva. Quarta Edição. 740p. 2007.&96&Caracterização Ambiental I&9788536308845&2&complementar\\
12&livro&Santos, R.F. Planejamento Ambiental: teoria e prática. São Paulo:Oficina de Textos. 184 p. 2004.&96&Caracterização Ambiental I&978-85-86238-62-8&1&complementar\\
14&livro&Townsend, C.R.; Begon, M. Harper, J.L. Fundamentos em Ecologia. Artmed Editora. Porto Alegre, RS. 576p. 2010.&80&Ecologia&9788536320649&10&basica\\
15&livro&Ricklefs, R.E. A Economia da Natureza. Editora Guanabara Koogan, Rio de Janeiro, RJ, sétima edição. 503p. 2016&80&Ecologia&9788527729628&5&basica\\
16&livro&Cullen Jr., L.; Rudran, R.; Valladares-Pádua, C. (orgs). Métodos de Estudos em Biologia da Conservação e Manejo da Vida Silvestre. Editora UFPR, segunda edição revisada. 652p. 2006.&80&Ecologia&978-8573351743&10&basica\\
17&livro&Cain, M.; Bowman, W.D.; Hacker, S.D. Ecologia. Artmed Editora. Porto Alegre, RS. 644p. 2011.&80&Ecologia&9788536325477&5&basica\\
18&livro&Harper, J.L. Ecologia: De indivíduos a Ecossistemas. Editora Saraiva. Quarta Edição. 740p. 2007.&80&Ecologia&9788536308845&5&basica\\
19&livro&Gotelli, N.J.; Ellison, A.M. Princípios de Estatística em Ecologia. Artmed Editora. Porto Alegre, RS. 532p. 2010.&80&Ecologia&9788536324326&3&complementar\\
20&livro&Odum, E.P. Fundamentos em Ecologia. Fundação  Calouste Gulbenkian. Lisboa, Portugal, sétima edição. 928p. 2004.&80&Ecologia&9789723101584&1&complementar\\
21&livro&Santos, R.F. Planejamento Ambiental: teoria e prática. São Paulo:Oficina de Textos. 184 p. 2004.&80&Ecologia&978-85-86238-62-8&1&complementar\\
22&livro&Instalacoes eletricas industriais. Mamede Filho, Joao. LTC Ed., 7.ed. de acordo com a nova norma brasileira, NBR 5410:2004 e 14.039. 2009. CDD: 621.31924&135&Eletricidade aplicada e Instrumentação Ambiental&978852161520-0&10&basica\\
23&livro&Instalações elétricas. Creder, Helio.  LTC,  15. ed. 2007. CDD: 621.31042&135&Eletricidade aplicada e Instrumentação Ambiental&9788521615675&10&basica\\
24&livro&Instalações elétricas. Niskier, Julio.  LTC, 5. ed. 2008. CDD: 621.31042&135&Eletricidade aplicada e Instrumentação Ambiental&9788521615897&10&basica\\
25&livro&TEIXEIRA, W.; TOLEDO, M. C. M.; FAIRCHILD, T. R.; TAIOLI, F. Decifrando a Terra. 2. ed. São Paulo : Oficina de Textos, 2009. 624 p.&132&Geologia&978-85-040-1439-6&10&basica\\
26&livro&FITTS, C.R. Águas subterrâneas. 2. ed. Rio de Janeiro : Elsevier, 2015.&132&Geologia&978-85-352-7744-9&5&basica\\
27&livro&HASUI, Y; CARNEIRO, C.D.R.; ALMEIDA, F.F.M; BARTORELLI, A. Geologia do Brasil. 1. ed. São Paulo: Editora Beca, 2013. 900 p.&132&Geologia&978-85-627-6810-1&5&basica\\
28&livro&CHIOSSI, N.H. Geologia aplicada à engenharia. 3. ed. São Paulo: Oficina de Textos, 2013. 424p.&132&Geologia&978-85-797-5083-0&10&complementar\\
29&livro&POPP, J. H. Geologia Geral. 6. Ed. Rio de Janeiro : Livros Técnicos e Científicos, 2010. 324p.&132&Geologia&978-85-216-1760-0&8&complementar\\
30&livro&QUEIROZ, RUDNEY C. Geologia e geotecnia básica para a engenharia civil. 1. ed. São Paulo : Bluncher, 2016. 416p.&132&Geologia&978-85-212-0956-0&3&complementar\\
31&livro&FEITOSA, F. A. C., J. M. FILHO. Hidrogeologia: Conceitos e Aplicações. 3. ed. rev. e ampl. Rio de Janeiro : CPRM; Recife: LABHID, 2008. 812 p.&132&Geologia&978-85-749-9061-3&3&complementar\\
32&livro&LEINZ, V.; AMARAL, S.C. Geologia Geral. 14. ed. São Paulo : Companhia Editora Nacional, 2003. 400p.&132&Geologia&978-85-040-0354-3&1&complementar\\
33&livro&CHRISTOFOLETTI, A. Geomorfologia. 2. ed. São Paulo : Blucher, 1980. 188p.&82&Geomorfologia&978-85-212-0130-4&10&basica\\
34&livro&TEIXEIRA, W.; TOLEDO, M. C. M.; FAIRCHILD, T. R.; TAIOLI, F. Decifrando a Terra. 2. ed. São Paulo : Oficina de Textos, 2009. 624 p.&82&Geomorfologia&978-85-040-1439-6&5&complementar\\
35&livro&GUIDICINI, G; NIABLE, C.N. Estabilidade de Taludes Naturais e de Escavação. 2. ed. São Paulo: Blucher, 1984. 216 p.&82&Geomorfologia&978-85-212-0186-9&5&complementar\\
36&livro&FLORENZANO, T. G. Geomorfologia: Conceitos e tecnologias atuais. 1. ed. São Paulo : Oficina de Textos, 2008. 320 p.&82&Geomorfologia&978-85-862-3865-9&5&basica\\
37&livro&GUERRA, A.T.J. Geomorfologia – uma atualização de bases e conceitos. 9. ed. Rio de Janeiro: Bertrand Brasil, 2012. 472 p.&82&Geomorfologia&978-85-286-0326-2&5&basica\\
38&livro&ASHBY,M.F. Engenharia ambiental: conceitos, tecnologia e gestão. 1. ed. Rio de Janeiro : Elsevier, 2013. 789 p.&82&Geomorfologia&978-85-352-5954-4&3&complementar\\
39&livro&GUERRA, A. J. T; JORGE, M. C. O. Processos erosivos e recuperação de áreas degradadas. 1. ed. São Paulo: Oficina de Textos, 2013. 192 p.&82&Geomorfologia&978-85-797-5079-3&3&complementar\\
40&livro&PINTO, C. S. Curso Básico de Mecânica dos Solos. 3. ed. São Paulo: Oficina de Textos, 2006. 367 p.&82&Geomorfologia&978-85-862-3851-2&4&complementar\\
42&outros&CASSETI, V. Geomorfologia. [S.l.]: [2005]. Disponível em: <http://www.funape.org.br/geomorfologia/>. Acesso em: . 17/01/2017.&82&Geomorfologia&&&outro\\
44&livro&D’ALMEIDA, Maria Luiza Otero, VILHENA, André. Lixo Municipal: Manual de Gerenciamento Integrado. São Paulo: IPT/CEMPRE, 2000.&118&Projeto e Gestão dos Resíduos Sólidos&&3&basica\\
45&livro&&75&Física I&978-85-216-1903-l&8&basica\\
46&livro&OLIVEIRA, J.B. Pedologia aplicada.  3ed. Piracicaba: FEALQ, 2008.&81&Solos&978-85-7133-064-1&8&basica\\
47&livro&MELO, V. F.; ALLEONI, L. R. F. (editores). Química e mineralogia do solo. Vol 1.Viçosa – MG: SBCS. 2009&81&Solos&978-85-86504-04-4&8&basica\\
48&livro&MELO, V. F.; ALLEONI, L. R. F. (editores). Química e mineralogia do solo. Vol 2.Viçosa – MG: SBCS. 2009&81&Solos&978-85-86504-05-1&8&basica\\
49&livro&LEPSCH, I. F. Formação e Conservação do solo. São Paulo: Oficina de Textos, 2002&81&Solos&978-85-7975-008-3&5&complementar\\
50&livro&REICHARDT, K; TIMM, L. C. Solo, planta e atmosfera: conceitos, processos e aplicações. Barueri/SP: Manole, 2004.&81&Solos&978-85-204-1773-6&5&complementar\\
51&livro&BRADY, N. C. Natureza e propriedades dos solos. Trad. Antônio B. N. Figueiredo. 7ed. Rio de Janeiro: Freitas Bastos, 1989.&81&Solos&&2&complementar\\
52&livro&GLIESSMAN, S. R. Agroecologia: processos ecológicos em agricultura sustentável. 2ed. Porto Alegre: Ed.Universidade/UFRGS. 2001&81&Solos&978-85-7025603-5&2&complementar\\
53&livro&Guerra, A.J.T., Silva, A.S., Garrido, R.M.B. Erosão e Conservação dos Solos - Conceitos, Temas e Aplicações. 8. ed. Rio de Janeiro: Bertrand Do Brasil, 2012.&109&Manejo e Conservação dos Recursos Naturais&9788528607383&5&basica\\
55&livro&Barbosa, R.P. Recursos Naturais E Biodiversidade: Preservação E Conservação Dos Ecossistemas - Série Eixos. São Paulo: Saraiva. 2014&109&Manejo e Conservação dos Recursos Naturais&978-85-3650-870-2&5&basica\\
56&livro&Schubart, H.O.R. Parte 3: Gestão de Recursos Hídricos e Gestão do Uso do Solo: O Zoneamento Ecológico-Econômico E A Gestão Dos Recursos Hídricos. Em: Interfaces da Gestão de Recursos Hídricos. Ed. 2000. Disponível em: http://www.uff.br/cienciaambiental/biblioteca/rhidricos/parte3.pdf&109&Manejo e Conservação dos Recursos Naturais&&5&complementar\\
57&livro&Mota, F.S. Preservação e conservação dos recursos hídricos. 2ª. Edição. Rio de Janeiro: ABES. 1995.&109&Manejo e Conservação dos Recursos Naturais&978-8570-221-18-5&5&complementar\\
58&livro&MELO, Itamar Soares e AZEVEDO, João Lucio. MICROBIOLOGIA AMBIENTAL - 2/ED. Editora: EMBRAPA - 2008&92&Microbiologia Ambiental&9788585771447&&basica\\
59&livro&MICHAEL J. PELCZAR JR., E.C.S. CHAN, NOEL R. KRIEG. Microbiologia: conceitos e aplicações. Vol 1 Editora: Pearson Education - 2ª – 1997&92&Microbiologia Ambiental&8534604541&&basica\\
60&livro&MICHAEL J. PELCZAR JR., E.C.S. CHAN, NOEL R. KRIEG. Microbiologia: conceitos e aplicações. Vol  2. Editora: Pearson Education - 2ª – 1997&92&Microbiologia Ambiental&8534604541&&basica\\
61&livro&MICROBIOLOGIA DE BROCK 14ª EDICAO C/CD ROM MICHAEL T. MADIGAN, JOHN M. MARTINKO, JACK PARKER - Editora Pearson / Prentice Hall (Grupo Pearson)&92&Microbiologia Ambiental&9788582712979&&complementar\\
62&livro&GERARD J. TORTORA; BERDELL R. FUNKE; CHRISTINE L. CASE - Editora Artmed MICROBIOLOGIA - 10ª EDIÇÃO - 2011&92&Microbiologia Ambiental&9788582713532&&complementar\\
63&livro&MANUAL DE MÉTODOS DE ANÁLISE MICROBIOLÓGICA DA ÁGUA NEUSELY DA SILVA/ROMEU C.NETO/VALÉRIA C.A. /NELIAN - Editora Varela&92&Microbiologia Ambiental&9788577590131&8&complementar\\
64&livro&MANUAL PRÁTICO DE MICROBIOLOGIA BÁSICA ROGERIO LACAZ RUIZ - Editora EDUSP&92&Microbiologia Ambiental&85-314-0540-8&8&complementar\\
65&livro&PRÁTICAS DE MICROBIOLOGIA VERMELHO, ALANE BEATRIZ - PEREIRA, ANTÔNIO FERREIRA - COELHO, ROSALIE REED RODRIGUES - SOUTO-PADRÓN, - Editora SonoPress RIMO&92&Microbiologia Ambiental&8527711656&8&complementar\\
66&livro&Ogenis Magno Brilhante e Luiz Querino de A. Caldas (Coords.) Gestão e Avaliação de Risco em Saúde Ambiental. ISBN 85-85676-56-6. 2a reimpressão: 2004. 1a reimpressão: 2002. (1a edição: 1999). 155p.  Editora Fiocruz&115&Saúde e Vigilância Ambiental&85-85676-56-6&5&basica\\
67&livro&Luís Sérgio Ozório Valentim. Requalificação urbana, contaminação do solo e riscos à saúde: um caso na cidade de São Paulo. Annablume, 2007. 159 P.&115&Saúde e Vigilância Ambiental&8574197246&1&complementar\\
68&livro&Cristina Lucia Silveira Sisinno, Rosália Maria de Oliveira (oRG.) Resíduos Sólidos, Ambiente e Saúde: uma visão multidisciplinar. 3ª reimpressão: 2006. 2ª reimpressão: 2003. 1ª reimpressão: 2002 (1ª edição: 2000). 138 P ISBN: 85-85676-80-9&115&Saúde e Vigilância Ambiental&85-85676-80-9&1&complementar\\
69&livro&Freitas CM, Porto MFS, Machado JMH, organizadores. Acidentes industriais ampliados: desafios e perspectivas para o controle e a prevenção. Rio de Janeiro: Editora Fiocruz; 2000.&115&Saúde e Vigilância Ambiental&85-85676-72-8. 2000.&1&complementar\\
70&livro&Minayo MCS, Miranda AC. Saúde e ambiente sustentável: estreitando os nós. Rio de Janeiro: Ed. Fiocruz/Abrasco; 2002.&115&Saúde e Vigilância Ambiental&85-85471-11-5&1&complementar\\
71&livro&Papini S. Vigilância em Saúde Ambiental - Uma Nova Área da Ecologia.2ª ed. revista e ampliada. Rio de Janeiro: Editora Atheneu; 2012&115&Saúde e Vigilância Ambiental&9788538802198&5&basica\\
72&livro&Philippi Junior, Arlindo. Saneamento, saude e ambiente, fundamentos para um desenvolvimento sustentavel. ISBN: 8520421881 ISBN-13: 9788520421888 Editora: MANOLE&115&Saúde e Vigilância Ambiental&8520421881 (ISBN-13: 9788520421888)&5&basica\\
73&livro&AKERMAN, Marco . Saúde e desenvolvimento local: princípios, conceitos, práticas e cooperação técnica. São Paulo: Hucitec, 2005.&115&Saúde e Vigilância Ambiental&9788527106856&5&complementar\\
74&livro&Rossetti, José Paschoal, Introdução à Economia. Editora Atlas, 992 pg, 2003&125&Economia Ambiental&9788522434671&20&basica\\
75&livro&Stiglitz, J. E. Introdução à Microeconomia. 2003&125&Economia Ambiental&9788535210446&20&basica\\
76&livro&Moraes, O. J. Economia Ambiental - Instrumentos Econômicos para o Desenvolvimento Sustentável. Editora Centauro. 224pg. 2009&125&Economia Ambiental&9788579280030&20&complementar\\
77&livro&Huberman, L. História da Riqueza do Homem. Editora LTC. 2010&125&Economia Ambiental&9788521617341&20&complementar\\
78&livro&Motta, R. S. Economia Ambiental. Editora FGV. 399 pg. 2008.&125&Economia Ambiental&978-8522505449&20&complementar\\
79&livro&Hirschfeld, H. Engenharia Econômica e Analise de Custos. Editora Atlas. 519 pg. 2001.&125&Economia Ambiental&8522426627&20&basica\\
80&livro&NBR ISO 14001:2015 . Sistemas de gestão ambiental — Requisitos com orientações para uso&150&Sistemas de Gestão Ambiental&&20&basica\\
81&outros&NBR ISO 14004:2005 Versão Corrigida 2:2007  Sistemas de gestão ambiental - Diretrizes gerais sobre princípios, sistemas e técnicas de apoio&150&Sistemas de Gestão Ambiental&&20&basica\\
82&outros&NBR ISO 19011:2012  Diretrizes para auditoria de sistemas de gestão&150&Sistemas de Gestão Ambiental&&20&basica\\
83&livro&Moura, L.A.A. Qualidade e Gestão Ambiental - Sustentabilidade e ISO 14.001.  Editora Del Rey. 418 pg. 2011.&150&Sistemas de Gestão Ambiental&9788538401766&40&basica\\
84&livro&Seiffert, M. E. B. Sistemas de Gestão Ambiental (SGA - ISO 14001). Editora Atlas. 168 pg. 2011&150&Sistemas de Gestão Ambiental&9788522462612&20&complementar\\
85&livro&Seiffert, M. E. B. ISO 14001 Sistemas de Gestão Ambiental - Implantação Objetiva e Econômica. Editora Atlas. 256 pg. 2011.&150&Sistemas de Gestão Ambiental&9788522461523&40&basica\\
86&outros&NBR ISO 14021:2013  Rótulos e declarações ambientais - Autodeclarações ambientais (Rotulagem do tipo II)&150&Sistemas de Gestão Ambiental&&20&complementar\\
87&outros&NBR ISO 14031:2015  Gestão ambiental - Avaliação de desempenho ambiental - Diretrizes&150&Sistemas de Gestão Ambiental&&&complementar\\
88&livro&Introdução ao Gerenciamento dos Recursos Hídricos&134&Gestão dos Recursos Hídricos&&0&basica\\
89&juridico&Política Nacional de recursos Hídricos Lei 9433/1997&134&Gestão dos Recursos Hídricos&&0&basica\\
90&livro&A evolução da gestão dos recursos hídricos no Brasil&134&Gestão dos Recursos Hídricos&&0&basica\\
91&livro&Experiências de Gestão dos Recursos Hídricos&134&Gestão dos Recursos Hídricos&&0&basica\\
92&livro&Brasil: A gestão da qualidade da água&134&Gestão dos Recursos Hídricos&&&basica\\
93&livro&Manual de Outorga de uso da água do Estado do Tocantins&134&Gestão dos Recursos Hídricos&&&complementar\\
94&juridico&Decreto 2432/2005 - regulamenta a outorga de uso da água no Estado do Tocantins&134&Gestão dos Recursos Hídricos&&&complementar\\
95&juridico&Lei 1307/2002 Política Estadual de recursos Hídricos - Tocantins&134&Gestão dos Recursos Hídricos&&&complementar\\
96&livro&JUNQUEIRA, LCU. 8a ed. Biologia celular e molecular. Guanabara Kooogan S. A. RJ. 332p. 2005.&126&Biologia&&15&basica\\
97&livro&RUPPERT, EE. Zoologia dos invertebrados. Roca. SP. 1145p. 2005.&126&Biologia&&15&basica\\
98&livro&Orr, Robert T., Biologia dos vertebrados / 5.ed. São Paulo, SP : Roca, 1986 x,508 p.&126&Biologia&&10&basica\\
99&livro&SADAVA, D.; RENARD, DG, BONAN, CD. Vida: a ciência da Biologia. 8a ed. Vol. 1. Artmed. Porto Alegre, RS. 2009.&126&Biologia&&20&basica\\
100&livro&AVERSI-FERREIRA, TA, Biologia celular e molecular / 2. ed. rev. e ampl. Campinas, SP: Átomo, 2013. 262 p.:&126&Biologia&&15&complementar\\
101&livro&SADAVA, D.; RENARD, DG, BONAN, CD. Vida: a ciência da Biologia. 8a ed. Vol. 2. Artmed. Porto Alegre, RS. 2009.&126&Biologia&&15&complementar\\
102&livro&SADAVA, D.; RENARD, DG, BONAN, CD. Vida: a ciência da Biologia. 8a ed. Vol. 3. Artmed. Porto Alegre, RS. 2009.&126&Biologia&&15&complementar\\
103&livro&LAKATOS, E. M., MARCONI, M. A. Fundamentos de metodologia científica. 7ed. São Paulo: Atlas. 270p. 2010&122&Trabalho de Conclusão de Curso I&&15&basica\\
104&livro&OLIVEIRA NETTO, ALVIM ANTONIO DE., Metodologia da Pesquisa Científica :guia prático para apresentação de trabalhos acadêmicos / 3.ed. rev. e atual. São Paulo : Visual Books, 2008. 192 p.&122&Trabalho de Conclusão de Curso I&&15&basica\\
105&livro&LAKATOS, E. M., MARCONI, M. A. Metodologia científica. 6ed. São Paulo: Atlas. 314p. 2011.&122&Trabalho de Conclusão de Curso I&&15&basica\\
106&livro&LAKATOS, E. M., MARCONI, M. A. Metodologia científica. 6ed. São Paulo: Atlas. 314p. 2011.&128&Trabalho de Conclusão de Curso II&&15&basica\\
107&livro&SEVERINO, Antônio Joaquim. Metodologia do trabalho científico. Cortez editora, 2014.&122&Trabalho de Conclusão de Curso I&9788524924484&15&complementar\\
108&livro&SEVERINO, Antônio Joaquim. Metodologia do trabalho científico. Cortez editora, 2014.&128&Trabalho de Conclusão de Curso II&&15&complementar\\
111&livro&OLIVEIRA NETTO, ALVIM ANTONIO DE., Metodologia da Pesquisa Científica :guia prático para apresentação de trabalhos acadêmicos / 3.ed. rev. e atual. São Paulo : Visual Books, 2008. 192 p.&128&Trabalho de Conclusão de Curso II&&15&basica\\
112&livro&LAKATOS, E. M., MARCONI, M. A. Fundamentos de metodologia científica. 7ed. São Paulo: Atlas. 270p. 2010&128&Trabalho de Conclusão de Curso II&&15&basica\\
113&livro&BOSCOV, M.E.G. Geotecnia Ambiental. 1. Ed. São Paulo : Oficina de Textos, 2008. 248 p.&103&Geotecnia Ambiental&978-85-862-3873-4&8&basica\\
114&livro&ZUQUETTE, L.V. Geotecnia Ambiental. 1. Ed. São Paulo : Elsevier , 2015. 432 p.&103&Geotecnia Ambiental&978-85-352-8058-6&10&basica\\
115&livro&OLIVEIRA, A. M.S., BRITO, S.N.A. Geologia de Engenharia. São Paulo : ABGE, 1998. 587 p.&103&Geotecnia Ambiental&978-85-727-000-2&5&basica\\
116&livro&GIAMPÁ, C.E.Q.; GONÇALVES, V. G. Águas Subterrâneas e Poços Tubulares Profundos. 2. ed. São Paulo : Oficina de Textos, 2013. 496 p&103&Geotecnia Ambiental&978-85-797-5086-1&5&complementar\\
117&livro&QUEIROZ, RUDNEY C. Geologia e Geotecnia Básica para Engenharia Civil. 1. ed. São Paulo : Bluncher, 2016. 416 p.&103&Geotecnia Ambiental&978-85-212-0956-0&3&complementar\\
118&livro&BATES, J.  Barragens de Rejeitos. São Paulo: Signus Editora, 2002. 112 p.&103&Geotecnia Ambiental&978-85-847-8031-6.&2&complementar\\
119&livro&MORAES, S.L.; TEIXEIRA, C.E.; MAXIMIANO, A.M.S. Guia de elaboração de planos de intervenção para o gerenciamento de áreas contaminadas. 1. ed. rev. São Paulo : IPT, 2014.395 p. Disponível em: https://www.google.com.br/\#q=GERENCIAMENTO+DE+\%C3\%81REAS+CONTAMINADAS+IPT. Acesso em: 17/01/2017.&103&Geotecnia Ambiental&978-85-09-00179-7&&complementar\\
120&livro&SCHIANETZ, B. Passivos Ambientais: levantamento histórico da periculosidade: ações de recuperação. Curitiba : SENAI-PR, 1999&103&Geotecnia Ambiental&&2&complementar\\
121&livro&BAIRD, C.; CANN, M. Química ambiental. 4ed. Porto Alegre :Ed. Bookman. 2011. 844p.&97&Química Ambiental&9788577808489&8&basica\\
122&livro&SPIRO, T. G.; STIGLIANI, W. M. Química ambiental. 2ed. Ed. São Paulo: Prentice Hall. 2008. 352p.&97&Química Ambiental&9788576051961.&8&basica\\
123&livro&RANGEL, M. B. A.; NOWACKI, C. C. B. Química ambiental: conceitos, processos e estudo dos impactos ao meio ambiente. São Paulo : Ed. Érica. 2014. 136p.&97&Química Ambiental&9788536509051&8&basica\\
124&livro&ROCHA, J. C.; ROSA, A. H.; CARDOSO, A. A. Introdução à química ambiental. 2ed.Porot Alegre : Ed. Bookman. 2009. 256p.&97&Química Ambiental&9788577804696.&8&complementar\\
125&livro&LEITE, A.E. Recursos Energéticos e Ambiente. Curitiba : Ed. Intersaberes. 2015. 320p.&107&Recursos Energéticos I&978-8544301449&15&basica\\
126&livro&JUNIOR, A.P. Matrizes Energéticas: Conceitos e Usos em Gestão e Planejamento. Barueri : Ed. Manole. 2010. 204 p.&107&Recursos Energéticos I&978-8520430385&15&basica\\
127&livro&RIBEIRO, J. A. Recursos Naturais como Insumo Energético. Um Estudo do Uso da Biomassa Florestal. Curitiba : Ed. Appris. 2016. 97 p.&107&Recursos Energéticos I&978-8581929682&15&basica\\
128&livro&BOBIN, J. A Energia. Ed. Instituto Piaget.  1998. 140 p.&107&Recursos Energéticos I&978-9727711864&15&complementar\\
129&livro&ATKINS, P.; Jones, L.; Princípios de Química – Questionando a Vida Moderna e o Meio Ambiente, 5º Ed.; São Paulo: Editora Bookman. 2011.&106&Química Geral&85-363-0668-8&22&basica\\
130&livro&BROWN, T. L., LeMay, H. E., Bursten, B. E. Burdge, J. R.; Química a Ciência Central. 9 ed. São Paulo: Editora Makron Books do Brasil. 2005.&106&Química Geral&85-87918-42-7&22&basica\\
131&livro&BRADY, J. E.; Senese, F. Química – a Matéria e suas Transformações, 5º Edição, Volume 1 e 2, Editora LTC, 2009.&106&Química Geral&978-85-216-1721.1 (v.2); 978-85-216-1720.4 (v.1)&22&basica\\
132&livro&KOTZ, J. C.; Treichel, P.M.; Weaver, G.C. Química Geral e Reações Químicas, Vol. 1 e 2, Editora Cengang Learning, Tradução da 6º Edição Norte Americana, 2009.&106&Química Geral&978-85-221-0691-2&15&complementar\\
133&livro&BROWN, L. S.; HOLME, T. A.; Química Geral Aplicada à Engenharia, Editora Cengage Learning, 2009.&106&Química Geral&978-85-221-0688-2&15&complementar\\
134&livro&RUSSEL, J. B., Química Geral. Vol 1 e 2;  2 ed.; São Paulo; Editora Makron Books do Brasil; 1994.&106&Química Geral&&15&complementar\\
135&livro&HARRIS, D. C. Análise Química Quantitativa. 8ª Edição., Rio de Janeiro, Editora  Livros Técnicos e Científicos S. A., 2012.&78&Química Analítica&978-85-216-2042-6&22&basica\\
136&livro&SKOOG, D. A.; West, D. A.; Holler, F. J.; Crouch, S. R.; Fundamentos de Química Analítica, 9ª Edição, Editora Thomson, 2014.&78&Química Analítica&978-85-221-1660-7&22&basica\\
137&livro&VOGEL, A. I. Analise Química Quantitativa. 6 ed. Rio de Janeiro: Livros Técnicos e Científicos. 2002.&78&Química Analítica&&15&complementar\\
138&livro&SOLOMONS, T. W. G., FRYHLE, C. B.; Química Orgânica, Editora LTC, Vol 1, 10ª Edição, Rio de Janeiro, 2012.&87&Química Orgânica&&22&basica\\
139&livro&SOLOMONS, T. W. G., FRYHLE, C. B.; Química Orgânica, Editora LTC, Vol 2, 10ª Edição, Rio de Janeiro, 2012.&87&Química Orgânica&&22&basica\\
140&livro&CARREY, F. A.; Química Orgânica, 7º Edição, Editora Bookman, Vol. 1, 2011.&87&Química Orgânica&978-85-8055-053-5&22&basica\\
141&livro&CARREY, F. A.; Química Orgânica, 7º Edição, Editora Bookman, Vol. 2, 2011.&87&Química Orgânica&978-85-63308-89-4&22&basica\\
142&livro&BRUICE, P. Y.; Química Orgânica, Editora Pearson Prentice Hall, Vol 1, 4º Edição, 2006.&87&Química Orgânica&&22&basica\\
143&livro&BRUICE, P. Y.; Química Orgânica, Editora Pearson Prentice Hall, Vol 2, 4º Edição, 2006.&87&Química Orgânica&&22&basica\\
144&livro&McMURRY, J.; Química Orgânica COMBO, Tradução da 6º Edição Norte-Americana, Editora Thomson, 2005.&87&Química Orgânica&&10&complementar\\
145&livro&Martland, Carl D. Avaliação de Projetos: Por uma Infraestrutura Sustentável. LTC:São Paulo. 1014. 424p.&129&Elaboração de Projetos em Engenharia&9788521624004&&basica\\
146&livro&ANDERSON, PHILIPKORTSCHOT, MARK T.MCCAHAN, SUSAN. PROJETOS DE ENGENHARIA - UMA INTRODUÇAO. LTC: São Paulo. 2014&129&Elaboração de Projetos em Engenharia&521634455&&basica\\
147&outros&O que é Gerenciamento de Projetos? disponível em:<https://brasil.pmi.org/brazil/AboutUs/WhatIsProjectManagement.aspx>. Acesso em: 14 maio 2018.&129&Elaboração de Projetos em Engenharia&&&outro\\
148&livro&Brunetti, Franco., Mecânica dos fluidos / 2.ed.rev. São Paulo : Prentice Hall, 2008. 431 p&90&Fenômenos de Transporte&9788581433035&11&basica\\
149&livro&BRASIL. Fundação Nacional de Saúde. Manual de Saneamento. 3a ed. Brasília: Fundação Nacional de Saúde, 2016. 408 p.&118&Projeto e Gestão dos Resíduos Sólidos&&&basica\\
150&juridico&BRASIL. Lei 12305. Política Nacional de Resíduos Sólidos&118&Projeto e Gestão dos Resíduos Sólidos&&1&basica\\
151&livro&BRAGA, B. et al. Introdução a Engenharia Ambiental. São Paulo: Ed. Prentice Hall, 2002.&144&Introdução A Engenharia Ambiental&&1&basica\\
152&livro&MOTA, S. Introdução à Engenharia Ambiental. Rio de Janeiro: Ed. ABES, 2000.&144&Introdução A Engenharia Ambiental&&1&basica\\
153&livro&MONTEIRO, J. H. P. et al. Manual de gerenciamento integrado de resíduos sólidos. Rio de Janeiro, IBAM, 2001. 200 p. Disponível em: < www.web-resol.org/cartilha4/manual.pdf>. Acesso em 30 ago. 2010.&118&Projeto e Gestão dos Resíduos Sólidos&&&basica\\
154&juridico&ASSOCIAÇÃO BRASILEIRA DE NORMAS TÉCNICAS-ABNT. NBR 13896 - Aterros de resíduos não perigosos: critérios para projeto, implantação e operação. Rio de Janeiro, 1997.&118&Projeto e Gestão dos Resíduos Sólidos&&1&complementar\\
155&juridico&ASSOCIAÇÃO BRASILEIRA DE NORMAS TÉCNICAS-ABNT. NBR 8419 - Apresentação de projetos de aterros sanitários de resíduos sólidos urbanos - Procedimento. Rio de Janeiro, 1996.&118&Projeto e Gestão dos Resíduos Sólidos&&&complementar\\
156&livro&PEREIRA NETO, João Tinôco. Manual de compostagem: processo de baixo custo. 1. ed. Viçosa, MG: Ed. da UFV, 2007. 81 p. (Soluções).&118&Projeto e Gestão dos Resíduos Sólidos&&&complementar\\
157&livro&BRASIL. Fundação Nacional de Saúde. Manual de Saneamento. 3a ed. Brasília: Fundação Nacional de Saúde, 2016. 408 p.&144&Introdução A Engenharia Ambiental&&1&complementar\\
158&livro&GUERRA, Antonio José Teixeira; CUNHA, Sandra Baptista da (Org.). Impactos ambientais urbanos no Brasil. 6.ed. Rio de Janeiro: Bertrand Brasil, 2010. 416 p.&144&Introdução A Engenharia Ambiental&9788528608021&&complementar\\
159&livro&FELLENBERG, Gunter. Introdução aos problemas da poluição ambiental. São Paulo, SP: EPU: 1980. xvi, 196 p.&144&Introdução A Engenharia Ambiental&8512490403&&complementar\\
160&livro&PINTO, W.D. "Legislação Federal de meio ambiente". IBAMA, Brasília, 1996.&114&Direito Ambiental&&&complementar\\
161&livro&ANTUNES, Paulo de Bessa. Direito Ambiental. Rio de Janeiro: Lumen Juris.&114&Direito Ambiental&&1&basica\\
162&livro&BENJAMIN, Antonio Herman V (coord.). Dano ambiental: prevenção, reparação e repressão. São Paulo: Revista dos Tribunais.&114&Direito Ambiental&&&basica\\
163&livro&FIORILLO, Celso Antonio P. Curso de Direito Ambiental Brasileiro. São Paulo: Saraiva.&114&Direito Ambiental&&&basica\\
164&livro&MELCONIAN, S.. Mecânica técnica e Resistência dos Materiais. 9. ed., rev. atualizada. São Paulo: Érica. 2013&93&Resistência dos Materiais&&8&basica\\
165&livro&BERR, F.P.; JOHNSTON, E. R. Resistência dos Materiais. São Paulo, Makron Books, 2010. GARE, J. M. Mecânica dos Materiais. São Paulo: Pioneira Thomson Learning, 2011. 698 p.&93&Resistência dos Materiais&&8&basica\\
166&livro&HIBBELER, R. C.. Resistência dos Materiais. 3ª ed. Rio de Janeiro: LTC, 2000.&93&Resistência dos Materiais&&8&basica\\
167&livro&CHIAVERINI, V. (1986). Tecnologia Mecânica, vol. I, II e III, 2ª ed. Brasil, S. Paulo: Editora McGraw-Hill.&93&Resistência dos Materiais&&6&basica\\
168&livro&NASH, W.A. Resistência dos Materiais. São Paulo, Mc Graw Hill, 1982.&93&Resistência dos Materiais&&4&complementar\\
169&livro&TIMOSHENKO, S. P.. Mecânica dos sólidos. Rio de Janeiro: LTC. 1989.&93&Resistência dos Materiais&&4&complementar\\
170&livro&COLLINS, J. A.: Projeto Mecânico de Elementos de Máquinas: Uma Perspectiva de Prevenção da Falha. 1º Edição. LTC, 2006.&124&Noções Básicas de Máquinas e Equipamentos&&6&basica\\
171&livro&BEER, F, P.; JOHNSTON, R.: Mecânica Vetorial para Engenheiros. 7º Edição. Makron Books. 2006.&124&Noções Básicas de Máquinas e Equipamentos&&4&complementar\\
172&livro&OLIVEIRA, F. E. M. de. Estatística e Probabilidade. São Paulo: Ed. Atlas, 221p. 1999.&119&Planejamento Ambiental&&10&basica\\
173&livro&Metodos numericos para a resolução de problemas logicos. Del Picchia, Walter. E. Blucher, xii, 395 p. ;&139&Métodos Numéricos&UFT00049425&10&basica\\
174&livro&Conjuntos numericos e potencias. Dienes, Zoltan P. EPU,3.ed. rev.- 141 p.:&139&Métodos Numéricos&UFT00032568&10&basica\\
175&livro&Construção e propriedades de alguns conjuntos numéricos. Bastos, Evangelino de Oliveira. 63 f.&139&Métodos Numéricos&UFT00093836&10&basica\\
176&livro&Padrões numericos e sequências. Carvalho, Maria Cecilia Costa e Silva. Moderna, 79p.&139&Métodos Numéricos&UFT00068356&10&basica\\
177&livro&Calculo numerico. Franco, Neide Bertoldi. Pearson Prentice Hall. 505p.&139&Métodos Numéricos&UFT00069713&10&basica\\
178&livro&Calculo numerico :aspectos teoricos e computacionais. Ruggiero, Marcia A. Gomes.	Makron Books,	2.ed. -	406p. :&139&Métodos Numéricos&UFT00032917&10&basica\\
179&livro&Calculo numerico :(com aplicações)		Harbra,	2. ed. -	367 p. ;&139&Métodos Numéricos&UFT00049116&10&basica\\
180&livro&Cálculo numérico :características matemáticas e computacionais dos métodos numéricos	Sperandio, Décio.	Pearson Prentice Hall,		354p.;&139&Métodos Numéricos&UFT00079402&10&basica\\
181&livro&Instrumentacao industrial		Interciencia,	2 ed.	583 p.&135&Eletricidade aplicada e Instrumentação Ambiental&UFT00087969&10&basica\\
182&livro&Instrumentacao, controle e automacao de processos	Alves, Jose Luiz Loureiro.	LTC Ed.,		270p.&135&Eletricidade aplicada e Instrumentação Ambiental&UFT00067310&10&basica\\
183&livro&Circuitos experimentais em eletricidade e eletrônica	Tucci, Wilson J.	Nobel,		il.&135&Eletricidade aplicada e Instrumentação Ambiental&UFT00075461&10&basica\\
184&livro&Tsutiya, M.T. Além Sobrinho, P. Coleta e Transporte de Esgoto Sanitário. Departamento de Engenharia Hidráulica e Sanitária, Escola Politécnica da USP, 1999.&136&Projeto de Sistemas de Esgotamento Sanitário e de Tratamento de Águas Residuárias&&5&basica\\
185&livro&CAMPOS, J.R. (coord.). Tratamento de Esgoto Sanitário por Processo Anaeróbio e Disposição Controlada no Solo. RECOPE - PROSAB, São Carlos, SP, Brasil, 2001&136&Projeto de Sistemas de Esgotamento Sanitário e de Tratamento de Águas Residuárias&&5&basica\\
186&livro&JORDÃO, E. P. P., Tratamento de Esgotos Domésticos. 3ª Edição – Rio de Janeiro – ABES, 2010&136&Projeto de Sistemas de Esgotamento Sanitário e de Tratamento de Águas Residuárias&&5&basica\\
187&livro&DI BERNARDO, L.; DI BERNARDO, A. Métodos e Técnicas de Tratamento de Água. Volume I e II. Ed. RIMA. 2ª. Ed. 2005&146&Tratamento de Água&&5&basica\\
188&livro&LIBANIO, Fundamentos de Qualidade e Tratamento de Água. 3ª Ed. Belo Horizonte. 2017&146&Tratamento de Água&&5&basica\\
189&livro&DI BERNARDO, L., DANTAS, A D. B., VOLTAN, P. E. N. Tratabilidade de Água e dos resíduos gerados em estações de tratamento de água. São Carlos, LDiBe editora, 2011&146&Tratamento de Água&&5&basica\\
190&livro&DI BERNARDO, L.; DI BERNARDO, A. Métodos e Técnicas de Tratamento de Água. Volume I e II. Ed. RIMA. 2ª. Ed. 2005&105&Processos e Operações Unitárias Aplicados Na Engenharia Ambiental&&5&basica\\
191&livro&JORDÃO, E. P. P., Tratamento de Esgotos Domésticos. 3ª Edição – Rio de Janeiro – ABES, 2010&105&Processos e Operações Unitárias Aplicados Na Engenharia Ambiental&&5&basica\\
192&livro&AZEVEDO NETTO. Manual de Hidráulica. Ed. Edgard Blucher. 8a Edição. São Paulo/SP.1998.&123&Projeto de Sistemas de Abastecimento e de Tratamento de Água&&5&basica\\
193&livro&TSUTIYA, M.T. Abastecimento de Água. 1ª Edição. Departamento de Engenharia Hidráulica e Sanitária da Escola Politécnica da USP. 2004.&123&Projeto de Sistemas de Abastecimento e de Tratamento de Água&&5&basica\\
195&livro&MASSARI. VICTOR L. Gerenciamento Ágil de Projetos. Brasport:2014. 256p.&129&Elaboração de Projetos em Engenharia&978-85-7452-696-6&11&basica\\
196&livro&Keelling,Ralph e Branco, Renato Henrique Ferreira. Gestão de Projetos - Uma Abordagem Global - 3ª Ed. 2014. 288p.&129&Elaboração de Projetos em Engenharia&9788502227101&&basica\\
197&livro&Monteiro Teixeira, Júlio. Gestão Visual De Projetos - Utilizando A Informação Para Inovar. Saraiva: São Paulo. 2014. 209p.&129&Elaboração de Projetos em Engenharia&9788550801711&11&complementar\\
198&livro&Guerreiro, Fernando.FERRAMENTAS ESTRATÉGICAS NA GESTÃO DE PROJETOS. Saraiva:São Paulo. 2016. e-book.&129&Elaboração de Projetos em Engenharia&&11&complementar\\
199&livro&Guide to the Project Management Body of Knowledge (PMBOK® Guide). PMI. 6 ed. 2017&129&Elaboração de Projetos em Engenharia&&&complementar\\
200&livro&BORESI, A. P.; SCHMIDT, R. J. Estática. São Paulo: Pioneira Thomson Learning, 2003. 673 p.&89&Mecânica da Engenharia&&11&basica\\
201&&HIBBELER, R. C. Mecânica, Estática. 12. ed .. Rio de Janeiro: LTC Editora. 2011. 265p.&89&Mecânica da Engenharia&&11&\\
203&livro&GIACAGLIA, GIORGIO E. O. Mecânica Geral. 10ª ed. Rio de Janeiro: Campus. 1982.&89&Mecânica da Engenharia&&11&complementar\\
204&livro&HIGDON-STILES, D. et. al. Mecânica. Rio de Janeiro, Prentice-Hall, 1984.&89&Mecânica da Engenharia&&5&complementar\\
207&livro&SINGER, F. L.. Mecânica para engenheiros. 2.00 . São Paulo: Harbra. 1981&89&Mecânica da Engenharia&&5&complementar\\
208&livro&Beer, Ferdinand P.; Johnston, E. Russell;Eisenberg, Elliot R. Mecânica Vetorial Para Engenheiros - Estática - 9ª Ed. São Paulo: MacGrawHill, 2011&89&Mecânica da Engenharia&&11&basica\\
209&livro&FOX, Robert W; MCDONALD, Alan T. Introdução à mecânica dos. 5. ed. Rio de Janeiro : LTC, 2001. 504 p.&90&Fenômenos de Transporte&&11&basica\\
210&livro&GILES, Ranald V. Mecânica de fluidos e hidráulica. 2. ed. São Paulo : Makron, 1996. 460 p.&90&Fenômenos de Transporte&&11&basica\\
211&livro&MUNSON, Bruce R. Fundamentos da mecanica dos fluidos. São Paulo : E. Blücher, 1997. 412 p&90&Fenômenos de Transporte&&5&complementar\\
212&livro&CATTANI, Mauro S. D. Elementos de mecanica dos fluidos. Sao Paulo : E. Blucher, c1990. 155p.&90&Fenômenos de Transporte&&5&complementar\\
213&livro&ROMA, Woodrow N. L. Fenomenos de transporte para engenharia 2.ed. São Paulo: RiMa, 2006. 276 p.&90&Fenômenos de Transporte&&5&complementar\\
214&livro&Hickman Jr, C.P. et al. 16a. ed. 2016. Princípios Integrados de Zoologia. Guanabara Koogan, R.J.&126&Biologia&9788527729369&5&complementar\\
\end{longtable}

\end{document}
